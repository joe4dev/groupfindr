\chapter{Hinführung und Problemstellung}
\label{kap1_hinfuehrung_problemstellung}


Test Test In vielen Lebenssituationen werden durch verschiedenen Individuen Gruppen gebildet. Hierbei existieren in der Literatur zahlreiche Definitionen für eine Gruppe. Nach Bierhoff und Frey ist eine Gruppe “eine Mehrzahl von Personen, die miteinander direkt interagieren und sich gegenseitig beeinflussen.”  \citep[Vgl.][S.~638]{bierhoff_handbuch_2006}
\newline\newline
Dabei können verschiedene Arten von Gruppen unterschieden werden.\citep[Vgl.][S.~11ff.]{thomas_grundris_1991} Es kann zum Beispiel zwischen einer Primärgruppe, wie die der Familie und Sekundärgruppen in Form von Arbeitsgruppen in einem Unternehmen differenziert werden. Mit Hilfe der Gruppengröße werden dabei zwischen Klein- und Großgruppen unterschieden. Eine Kleingruppe umfasst dabei maximal 30 Mitglieder, die der Großgruppe fasst maximal 100 Personen, da ansonsten keine direkte Interaktion gewährleistet ist. 
Häufig ist ebenfalls die Unterscheidung zwischen formellen und informellen Gruppen zu finden. \citep[Vgl.][S.~48]{spies_organisationspsychologie_2010}
Formelle Gruppen bekommen dabei die Ziele und die Regeln für die Zusammenarbeit von aussen vorgeben, bspw. der Universität oder dem Unternehmen vorgegeben. Ebenso sind den jeweiligen Gruppenmitgliedern einer formellen Gruppen spezifische Pflichten und Verantwortungen übertragen worden. Bei informellen Gruppen ist dies nicht der Fall und der Gruppe werden auch keine Ziele bzw. Aufgaben von aussen übertragen, vielmehr erfolgt die Bildung der informellen Gruppe bspw. aufgrund gleicher Bedürfnisse (bspw. die Begeisterung für die gleiche Fussballmannschaft). 
\newline\newline
Nach \citet{dick_teamwork_2013} ist ein Team "`eine Gruppe von Menschen, die gemeinsam an der Erreichung geteilter Ziele arbeiten, dabei verschiedene Rollen übernehmen und die miteinander kommunizieren, um so ihre Anstrengungen erfolgreich koordinieren zu können."' \citep[S.~1]{dick_teamwork_2013} Somit könnte ein Team als Teilmenge einer Gruppe angesehen werden, der Übergang ist jedoch fliessend. Daher wird auch vom "`Gruppe-Team-Kontinuum"' gesprochen.\citep[Vgl.][S.~14]{brettel_erfolgreiche_2009} In der vorliegenden Arbeit werden die Begriffe Gruppe und Team jedoch synonym gebraucht.
\newline\newline
Jede Gruppe durchläuft dabei die folgenden Phasen nach dem Modell von Tuckmann.\citep[Vgl.][S.~26-28]{dick_teamwork_2013} In der "`Forming"' Phase erfolgt ein erstes Abtasten und Kennenlernen der einzelnen Gruppenmitgliedern untereinandern. Die zweiten Phase ("`Storming"') ist durch Machtkämpfe sowie Streitigkeiten über die Regeln und Ziele innerhalb der Gruppe gekennzeichnet. Anschliessend erfolgt in der "`Norming"' Phase die Festlegung gemeinsame Ziele und Regeln für die Zusammenarbeit. Im ursprünglichen Modell war die "`Performing"' Phase die letzte Phase im Modell, in der die gemeinsame Arbeit zur Erreichung der festgelegten Ziele im Vordergrund steht. Später wurde das Modell jedoch durch die "`Adjourning"' Phase ergänzt, in der formelle Abschluss der Gruppentätigkeit und die anschliessende Auflösung im Vordergrund steht.
\newline\newline
Im Rahmen dieser Arbeit soll die Bildungsprozess der Gruppe durch eine Web-Anwendung erleichtert werden. Somit ist die Anwendung in der "`Forming"' Phase einzuordnen. Für die Gruppenart existieren theoretisch keine Beschränkungen, allerdings ist das graphische User-Interface (GUI) nur für Kleingruppen geeignet. Durch die Autoren wird die Anzahl von vier Gruppenmitglieder als optimal angesehen, ab acht Teilnehmer ist bereits die kritische Gruppengrösse erreicht. Die Idee für diese Art der Anwendung entstand durch die Problemstellung der Gruppenfindung für ein praktisches Projekt im Rahmen der Web 2.0 Vorlesung an der Universität Zürich im Frühlingssemester 2015.
