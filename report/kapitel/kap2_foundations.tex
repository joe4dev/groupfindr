\chapter{Related Work}
\label{related_work}
\section{Benefits Management}
\label{benefits_management}

It was and still is difficult to identify and measure the benefits which could be realized with IS/IT-investments. The focus of many methodologies is on the technical delivery of the IS/IT investments, like introducing information systems. This side can also be called as the ``supply side''. The counterpart is called ``demand side'' and focuses more on possible business benefits and their realization \citep[p.80]{ward_benefits_2012}. One of the most widely used and cited approach of benefits management to close the gap between the supply and demand side is developed by the ``Information Research Center'' at the Cranfield School of Management \citep[p.555]{braun_benefits_2009}. Therefore this section will concentrate on this approach by using the term benefits management (BM). Explanations about other approaches can be found in \citet[p.61ff.]{banninger_benefits_2004}.
\newline\newline
According to \citet{ward2002} benefits management is ``the process of organizing and managing such that the potential benefits arising from the use of IS/IT are actually realized'' \citep[p.439]{ward2002}. It should be understood as an life-cycle process which for example emphasizes the benefits delivery and benefits review instead of only a technology delivery or technology audits. It is also driven by a business case and change management plan as well as on the education in exploitation of technology of all stakeholders effected by new or changed information systems. To realize all possible benefits it is not only necessary to allow the involved people to use the information system but also to encourage them to think about new possibilities in the changed working environment and try them out personally. If this is possible the acceptance will be higher and further benefits can be identified and realized more easier.\citep[p.9-16]{ward_benefits_2012}
\newline\newline
Before describing the five process steps of the benefits management the author would like to mention the definitions for information technology (IT) and information systems (IS), which are given by \citet{ward_benefits_2012}. ``IT refers to the technology on which information systems operate and run'' \citep[p.17]{ward_benefits_2012}. ``IS are the means by which people and organizations, utilizing technology, gather, process, store, use and disseminate information'' \citep[p.17]{ward_benefits_2012}. Alternative definitions and explanations about information systems as well as refinements about business intelligence (systems) will be given the section \ref{classification_challenges}.

\label{process_benefits_management}

As the benefit management process can enable a common understanding over the benefits of investments in IT and IS it is a prerequisite that there are strategies for the business, IS and IT already in place, which can be linked by using the benefits management process.\citep[p.21]{ward_benefits_2012} The iterative process of benefits management consists as mentioned before of five steps, see figure \ref{process_model_benefitmanagement}. 

\begin{figure}[h]
\centering
\includegraphics[width=0.9\columnwidth]{graphiken/process-modell-benefitmanagement.png}%
\caption{Process steps of benefits management \citep[p.442]{ward2002}}%
\label{process_model_benefitmanagement}%
\end{figure} 

In the first step ``identifying and structuring the benefits'' \citep[cf.][p.69-72]{ward_benefits_2012} all involved stakeholders have to agree on investment objectives. To do so it is necessary to write down a set of statements how the situation should be after the successful completion of the IS/IT investment. The next more concrete task would be to identify the business benefits and their corresponding owners. A business benefit is simply an advantage for a single stakeholder or group of employees or partners, which represent a stakeholder group. For each business benefit it is important to emphasize where the advantage should be realized. Only then it is possible to measure the benefit. If a benefit can't be measured in quantitative (increased sales volume) or qualitative (like customer satisfaction through faster delivery) way and if it isn't possible to define one or more benefit owners it shouldn't be on the output list of this first step. In addition, the results of this step you should also consist of an benefits dependency network, which lists the necessary changes to be able to realize the benefits and how the IS/IT investment support those. Changes can be differentiated into ongoing business changes, which ``are new ways of working that are required to ensure the desired benefits are realized'' \citep[p.72]{ward_benefits_2012} and one-time enabling changes that are prerequisite for achieving the business changes or that are essential to bring the system into effective operation within the organization \citep[p.72]{ward_benefits_2012}.
\newline\newline
The second step ``plan benefits realization'' \citep[cf.][p.75f.]{ward_benefits_2012} consists of building a complete benefits plan (based on the previous list and their dependencies) and a business case for the planned IS/IT investment. The benefits (realization) plan consists of a full description of the identified benefits including both types of changes. It should contain the corresponding owners as well as measures for each benefit. An example to visualize the dependencies is the benefits dependency network. An template for a benefits dependency network is shown in figure \ref{business_dependency_network}. The template shows two boxes. The left box consists of the enablers and changes which have to be done in order to gain the planned business benefits. An example for an IS/IT enabler can be a customer relationship management (CRM) system, which offers the possibility to automate marketing campaigns. An enabling change would be the successful training of the employees of the marketing department in the new system. The changes in the campaign process which have to be done so that the process could be automated are examples for business changes. The right box consists of the business benefits and the investment objectives. The reduced processing time for the execution could be considered as a business benefit. One of the investment objectives to buy or build an CRM system could be the reducing of the costs for marketing campaigns. Further more concrete examples can be found in \citet[p.118-123]{ward_benefits_2012}.

\begin{figure}[h]
\centering
\includegraphics[width=0.9\columnwidth]{graphiken/business_dependency_network.png}%
\caption{Business dependency network \citep[p.445]{ward2002}}%
\label{business_dependency_network}%
\end{figure}

In the third step of the benefits management process \citep[cf.][p.75f.]{ward_benefits_2012} the execution of planned benefits takes place, as part of the overall project plan. By carrying out the benefits plan it is especially important that all stakeholders fulfill their responsibilities to execute the enabling changes as a prerequisite for the business changes. During the execution it is likely that something changes or has to be changed. Every adjustment of the benefits plan has to be verified by taking the following questions into account: ``What is the effect on the benefits and our ability to achieve them?'' \citep[p.76]{ward_benefits_2012} It is also possible to discover new possible benefits during the execution of the benefits plan, which have to be evaluated.
\newline\newline
The reviewing and evaluation of the execution results build the fourth step \citep[cf.][p.78f.]{ward_benefits_2012}. As the name of the step indicates this step has two purposes. The first one is to check if the planned benefits are already achieved and if not which actions may be necessary to realize the planned benefits. The second part of the first purpose is to become aware of the realization of benefits, which weren't planned and to engage possible disbenefits. The second purpose is that the organization can learn why some benefits were realized, but others weren't. This knowledge could help to improve the realization of benefits with IS/IT-investments in the future. It was shown by \citet[cited in][p.78]{ward_benefits_2012} that especially this task is making the difference for a continuous success of benefits realization.
\newline\newline
After reviewing and evaluating the realization of the business benefits gained by IS/IT investments all involved persons should be encouraged in the last step of the benefits management process \citep[cf.][p.79f.]{ward_benefits_2012} to think about further business benefits. It often happens that many employees and stakeholders become aware of useful ideas about enhancements or adjustments by working in the new or changed environment. This could be the foundation for a new iteration of the whole process.


\section{Business Intelligence}
\label{business_intelligence}

There are existing many definitions and explanations for the term business intelligence. The upcoming section gives an overview of the classification of business intelligence and will also explain the main ongoing challenges in the BI context. The next section will introduce some design principles for BI platforms to overcome those ongoing challenges.

\subsection{Classification and ongoing challenges}
\label{classification_challenges}

For information systems are existing various definitions and explanations, one was given in section \ref{benefits_management}. 
An other definition would be that a information system consists of individuals and computers, which are creating and/or using information while they are connected through communication relations \citep[translated from][p.131]{hansen_wirtschaftsinformatik_2009}.  An application system (german: Anwendungssystem) is a more specific class of information systems, because it address a concrete business process and the corresponding data, but doesn't include the individuals \citep[translated from][p.326]{stahlknechthasenkamp2005}.
\newline\newline
The application systems can be divided into two parts: a) Operational systems to support daily work, which is oriented on closed transactions. Typical examples are accounting or purchasing. b) Analytical systems focus on the information providing for all levels of managers. The terms of management system (german: Führungssystem) \citep[p.331]{stahlknechthasenkamp2005} and analytical information system \citep{chamonigluchowski2006} are also used as synonyms for analytical systems.
\newline\newline
In the context of the analytical systems the term ``Business Intelligence (BI)'' was established since the mid 90s \citep[p.1]{kempermehannaunger2006}. The term itself was already used by \citet{luhn1958}. Often the term business intelligence is used as a generic term for various things \citep[cf.][p.5]{gluchowski2001} and can be characterized in three ways including different aspects, see figure \ref{aspects_bi} \citep[p.7f.]{kempermehannaunger2006}. In a narrow sense BI only consists of techniques and systems which support directly the decision making process, like online analytical processing (olap) and management information system (MIS) or executive information systems (EIS). Another understanding of BI would be the analytical interpretation, which includes besides OLAP, MIS and EIS also text mining, data mining, ad-hoc reporting and analytical oriented applications like analytical CRM, balance scorecard systems or planning systems.

\begin{figure}[H]
\centering
\includegraphics[width=0.9\columnwidth]{graphiken/einordnung-bi_english.png}%
\caption{Aspects of BI \citep[p.7]{gluchowski2001}}%
\label{aspects_bi}%
\end{figure}

In this paper BI is referred in the wide sense (third way of understanding), that means that it includes all systems and methods with a decision support character including the data extraction, transformation and staging of the data in a data warehouse. Further explanations can be found in \citet{gluchowski2001}. Another alternative definition with the a comparable meaning is given by \citet{howson_bi_2007} ``Business Intelligence allows people at all levels of an organization to access, interact with, and analyze data to manage the business, improve performance, discover opportunities and operate efficiently'' \citep{howson_bi_2007}. In the scientific literature and practice the term ``decision support systems'' can be found, which provide support for ``ad hoc knowledge needs, performing knowledge derivation or discovery, direct accessibility by their decision-making users, userspecific
customization of functionality and interfaces, and/or learning from prior
decisional experiences'' \citep[p.IX]{burstein_handbook_2008} In this paper only the term BI system or platform is used and it refers to a system which has a decision support character for all levels of managers. 
\newline\newline
BI is used in many successful companies and is sometimes essential for competing with other companies \citep{chaudhuri_overview_2011}. But even if BI systems are around for some years, there are still existing many challenges in the usage and further development. Some of the problems are mentioned by \citet{ganosr_von_2010} such as data quality or performance issues. Additionally it can be difficult to achieve a diffusion of business intelligence reports and the possibility for an innovative use to quickly react on e.g. changed customer preferences \citep{kretzer_designer_2015}.
\newline\newline
Another general challenge for all informations systems, but specifically for business intelligence systems, is the processing, analyzing and visualization of the fast growing amount of heterogenous and unstructed data as well the ``classical'' structured data \citep{chaudhuri_overview_2011, economist_data_2010, dhar_data_2013}.

\subsection{Design principles for BI platforms}
\label{design_principles}
\citet{kretzer_designer_2015} acknowledge the previous shortcomings and investigate the diffusion of reports and the innovative use of BI platforms. They refer to the diffusion of reports as ``the number of users who request a certain report at a certain frequency'' \citep[p.677]{kretzer_designer_2015}. They adopted the definition of innovative use ``as employee's usage of a BI platform in novel ways to support their work'' \citep[cited in][p.677]{kretzer_designer_2015}. By using the design science research approach the first developed meta-requirements, which are interfering with the diffusion of reports and the innovative use of a BI platform. In a second step they derived specific design principles, which should be considered in the creating or adjusting of BI platforms.
\newline\newline
The identified meta-requirement to increase the diffusion of a report consists of three parts.\citep[cf.][p.680f.]{kretzer_designer_2015} First, it should be visible which reports were used and in what kind of manner (infectiousness). Second, the visibility of a report (use) should be increased for potential new users, if those are sharing similar profiles (same job title in different countries etc.) with previous users of the report (role equivalence). The third part applies to users whose proximity is measured in terms of the number, length and strength of paths (social cohesion). According to this part it is reasonable to improve the visibility of a report for users, who are working in several projects together, even if there job profiles are different.
\newline\newline
Discovering the impediments to innovative use \citet{kretzer_designer_2015} developed the following meta-requirement to address an innovative use of a BI platform: ``In order to increase innovative use, a BI platform needs to allow experimentation within predefined boundaries while avoiding experimentation outside these boundaries.''\citep[p.681]{kretzer_designer_2015} 
\newline\newline
Based on these meta-requirements two design principles are proposed.
\begin{itemize}
\item ``DP1: In order to increase diffusion of a certain report, the BI platform
should recommend that report to a potential new user based
on (a) the potential user’s preferences, (b) the infectiousness of
prior users of that report, (c) the role equivalence between the potential
user and previous users of that report, and (d) the social
cohesion between the potential user and previous users of that report.''\citep[p.683]{kretzer_designer_2015} 
\item ``DP2: In order to increase a user’s innovative use of a BI platform,
the BI platform should provide a permanent sandbox to the user in
which the user can load the data from the BI platform, modify it,
and enrich it with data from external data sources.''\citep[p.684]{kretzer_designer_2015} 
\end{itemize}

According to the author it is important to acknowledge that the proposed design principles offer the possibility to balance between the stability of a platform, which is important for a diffusion of reports by increasing the trustfulness of results in the reports and the flexibility of a platform, which is required to compete in successful way with changing business variables, such as new marketing data or changed customer preferences. 

\chapter{Results and Discussion}
\label{bm_for_bi}

After describing the related work, which have be done in the areas of benefits management and business intelligence this chapter will concentrate on the results of a literature review about the acceptance of benefits management including mentioning meta-requirements and design principles for benefits management. After that the author is describing possibilities to increase the BM acceptance on one side and following design principles for BM on the other side by using BI platform elements as BM artefacts. 

\section{Acceptance of benefits management}
\label{acceptance_benefits_management}

\citet[p.8]{ward_benefits_2012} claims that the benefits management process is successfully used in over 100 companies worldwide, even though it was developed in 1996. \citet{hesselmann_not_2015} rising the question why the adoption rate isn't much higher by exploring the determinants for benefits management acceptance on an individual level and the contextual factors, which can predict their importance on these factors ex-ante. Using the theory of planned behavior (TPB) to build a BM acceptance model is according to \citet[p.589]{hesselmann_not_2015} the right choice because it is a well-researched theory and isn't focusing on technical artefacts. Figure \ref{bm_acceptance_model} shows the conceptual model of the used constructs and propositions.

\begin{figure}[h]
\centering
\includegraphics[width=0.9\columnwidth]{graphiken/bm-acceptance-model.png}%
\caption{Benefits management acceptance model \citep[p.595]{hesselmann_not_2015}}%
\label{bm_acceptance_model}%
\end{figure}

The model by \citet{hesselmann_not_2015} implies that the intention to use BM and the actual usage behavior is positively related with an higher acceptance rate for the benefits management process itself. Performance expectancy in this context means that employees have to believe that they can improve their job performance by participating in the BM process. The determinant ``outcome expectancy'' is positively associated with the intention to use BM if the employees can expect incentives by using BM methods and tools. Social norm or pressure can have two sources. A supervisor or formal authorities who have the power to compliment or punish the use of BM \citep[cited in][p.591]{hesselmann_not_2015}. The second source are other employees by stating a positive opinion about BM that can have a positive influence on the intention to use BM. As the last direct determinant to the intention to use BM on an individual level are the facilitating conditions. \citet{hesselmann_not_2015} defining the facilitating conditions ``as the degree to which individuals perceive that they have the necessary resources and that there is organizational support to facilitate the BM activites'' \citep[cited in][p.592]{hesselmann_not_2015}. The explanations of the remaining variables can be found in \citet[p.592f.]{hesselmann_not_2015}.

\section{Design principles of benefits management}
\label{designprinciples_benefits_management}

\citet{ahlemann_exploiting_2013} also found out that the realization of benefits management is rather difficult. Therefore they did an field study to identify meta-requirements for BM and derived design principles to fulfill those meta-requirements. In table \ref{meta_requirements_bm} all the identified meta-requirements are listed. MR4 and MR8 got the highest amount of observations. Therefore it is seems important that those meta-requirements should be met to increase the success of benefits management realizations.

\begin{table}[H]
 \begin{tabular}{|p{6cm}|p{9cm}|}
  \hline
\textbf{Meta-requirement}		& \textbf{Description}\\
    \hline  
MR1: A BM artifact should support the distribution of benefit accountabilities
among the business and IT department (19 observations).
& 
Informants agreed on the requirement that any BM artifact should be very clear
about the responsibilities of IT and business during benefits analysis, planning
and realization.\\
\hline
MR2: A BM artifact should help define clear accountabilities for benefits realization on the business
side, especially when several business units are involved (10 observations).
&
Informants reported that, in some situations, benefits realization depends on
various business departments’ contributions while only one business
departments enjoys the benefits. This negatively impacts stakeholders’
willingness to cooperate.\\
\hline
MR3: A BM artifact should motivate relevant stakeholders to engage in benefits realization (22 observations).
&
Many informants have experienced a lack of motivation on the side of the business stakeholders, limiting benefits realization success.\\
\hline
MR4: A BM artifact should support benefits planning and realization in line with corporate strategy and IT strategy (24 observations).
&
There was strong consensus among informants that BM should be regarded as a strategic activity closely related to strategy implementation. The defined strategic objectives should be in line with the planned benefits.\\
\hline
MR5: A BM artifact should create transparency with regard to the degree of benefits realization (20
observations).
&
Informants unanimously stated that one of the very first steps when introducing benefits management should be establishing transparency with regard to benefits realization.\\
\hline
MR6: A BM artifact should help and guide stakeholders during benefits realization (22
observations).
&
Stakeholders find it difficult to plan and realize benefits on an operative level, especially if BM was only recently introduced into the organization. There is significant insecurity regarding what needs to be done.\\
\hline
MR7: A BM artifact should account for changing environmental conditions and benefits volatility (14
observations).
&
Informants stated that there is often a long time span between benefits analysis and benefits realization. As a consequence, benefits become volatile and often take a new form.\\
\hline
MR8: A BM artifact should allow for making mistakes during benefits analysis, planning and realization and should foster organizational learning (24 observations).
&
BM is generally regarded as a demanding management discipline. Immature organizations and stakeholders who are responsible for benefits realization tend to make many mistakes when analyzing, structuring, and measuring benefits for the first time.\\
\hline
MR9: A BM artifact should overcome “departmental egoism” (20 observations).
&
BM only reaches its full potential when stakeholders overcome “departmental egoism” and commit themselves to IS investment success. Benefits can rarely be implemented through one organizational unit alone.\\
\hline
  \end{tabular}
  \caption{Meta-requirements for BM \citep[p.7f.]{ahlemann_exploiting_2013}}
	\label{meta_requirements_bm}
\end{table}


After identifying the mentioned meta-requirements \citet{ahlemann_exploiting_2013} derived design principles, which have to be followed to increase the realization success of BM itself. The design principles and correspondend meta-requirements can be found in table \ref{design_principles_bm}. It is important to consider the design priniciples DP3, DP5 und DP6, because for the corresponding meta-requirements the highest amount of observations were discovered.

\begin{table}[H]
 \begin{tabular}{|p{12cm}|p{2cm}|}
  \hline
\textbf{Principle}		& \textbf{Meta-requirement}\\
    \hline  
DP1: Establish an accountability framework for benefits analysis, planning and realization 
& 
MR1
\newline
MR2
\\
\hline

DP2: Define benefit-related goals and incentives 
& 
MR3
\\
\hline

DP3: Integrate benefits management with strategic planning processes
&
MR3
\newline
MR4
\newline
MR5
\\
\hline

DP4: Implement dedicated benefits planning and realization processes
&
MR5
\newline
MR6
\\
\hline

DP5: Establish a benefits change management
&
MR6
\newline
MR7
\newline
MR8
\\
\hline

DP6: Continuously refine and optimize benefits analysis and measurement
&
MR8
\\
\hline

DP7: Cultivate benefits-related cross-departmental collaboration and joint target-setting
&
MR9
\\
\hline

DP8: Foster thinking based on boundaryspanning cause-effect chains
&
MR9
\\
\hline


\hline
  \end{tabular}
  \caption{Design priniciples for BM \citep[p.8f.]{ahlemann_exploiting_2013}}
	\label{design_principles_bm}
\end{table}

\section{Using BI elements as BM artefacts to increase BM acceptance}
\label{bi_elements_higher_acceptance}

After introducing determinants for the acceptance of BM as well as design principles for the BM realization I would like to provide some arguments why it is seems convenient to maximize the acceptance of benefits management in general by investing in business intelligence platforms. The ongoing challenges in BI get a lot of attention and willingness to overcome from the business \citep{gartner_threetrends_bi} and IT \citep{gartner_cio_investments} departments. In this context the investments can be used to create or change BI-platforms by using the proposed design principles to address the meta-requirements for generative BI-platforms, which were mentioned in section \ref{design_principles}. 
\newline\newline
The contribution of the paper is the analysis if two elements of BI platform (recommendation agents and sandboxes) can fulfill meta-requirements for BM and are build according to the design principles for the successful realization of BM on one hand and if those BI elements are influencing the determinants for a BM acceptance and usage in a positive way on the other hand. The selection of meta-requirements, design principles as well as the acceptance determinants are based on the subjective opinion of the author. Therefore not all the meta-requirements, design principles and acceptance determinants are set into relation with the BI elements recommendation agent and sandbox.
\newline\newline
By using recommendation agents as an output element of following the design principle to increase the diffusion of reports, new information about the actual usage behavior is automatically generated and can be complemented with the individual reasons for the preferred use of this report. The commenting of the own usage behavior can be combined with some form of incentives. If the usage statistics and the additionally comments are considered as BM artifacts, they can be used to identify, plan, realize and improve the long-term benefits of a business intelligence platform. According to the opinion of the author this would also address the MR4 for BM and will be consistent with the DP3 for continuous and successful adoption of benefits management activities, as shown in table \ref{bi_elements_meta_requirement_bm}.
\newline\newline
If the business intelligence platform allows to use sandboxes it is much more easier for the users to try out new ideas in a very fast and easy way without worrying about possible mistakes which could be done, because they will affect only the user himself. With this working attitude the employees are encouraged to educate themselves in the exploitation of technology on the one hand and exploring new benefits in continuous and iterative process on the other hand. By considering the sandbox of the business intelligence platform as an benefits management artefact the following meta-requirements 7 and 8 as well as their derived design principles 5 and 6 for BM (see table \ref{bi_elements_meta_requirement_bm}) can be addressed successfully (according to the opinion of the author).


\begin{table}[H]
  \begin{tabular}{|p{4.8cm}|p{4.8cm}|p{4.8cm}|}
 \hline
\textbf{Element of BI platform and outputs as BM artefacts} & \textbf{Meta-requirement for BM} & \textbf{Design principle for BM}\\
\hline  
A Recommendation agent with the usage statistics and supplementary comments.
&
MR4: A BM artifact should support benefits planning and realization in line with corporate strategy.
&
DP3: Integrate benefits management with strategic planning processes for continuous and successful adoption of benefits management activities.
\\
\hline

Sandbox for exploitation of technology and identification of further possible benefits.
&
MR7: BM artifact should account for changing environmental  conditions and benefits volatility.
&
DP5: Establish a benefits change management\newline
DP6: Continuously refine and optimize benefits analysis and measurement
\newline
\\
&
MR8: A BM artifact should allow for making mistakes during benefits analysis, planning and realization and should foster organizational learning. 
&
DP6: Continuously refine and optimize benefits analysis and measurement
\\
\hline
\end{tabular}
\caption{Elements of a BI platform, their outputs as BM artefacts and the corresponding meta-requirements and design principles for BM (table by author)}
\label{bi_elements_meta_requirement_bm}
\end{table}

By using the recommendation agent of the business intelligence platform the users can expect to increase their job performance by reducing the search time for the needed report. If the writing of comments on the own user behavior is related to individual incentives the outcome expectancy is also positively influenced. With the visibility of the usage behavior of reports the social pressure increases for inactive users or users with a low acceptance rate so far. Additionally it seems that the use of a permanent sandbox as a part of business intelligence platform and a BM artefact has an positive influence on the determinants ``facilitating conditions'', because it provides working conditions in which it is allowed to experiment and to fail from time to time. This effects directly the intention to use BM and the actual use of BM (see section \ref{acceptance_benefits_management}). Therefore the three determinants ``job performance'', ``outcome expectancy'' and ``facilitating conditions'' of BM acceptance (see section \ref{acceptance_benefits_management}) are also influenced in positive way. The proposed relations are visualized in table \ref{elements_bi_acceptance}.

\begin{table}[H]
  \begin{tabular}{|p{7cm}|p{7cm}|}
 \hline
\textbf{Element of BI platform and outputs as BM artefacts} & \textbf{positively influenced determinants of BM acceptance}\\
\hline  
A Recommendation agent with the usage statistics and supplementary comments.
&
Performance expectancy\newline
Outcome expectancy
\\
\hline
Sandbox for exploitation of technology and identification of further possible benefits.
&
Facilitating conditions
\newline
\\
\hline

  \end{tabular}
   \caption{Elements of a BI platform, their outputs as BM artefacts and the positively influenced determinants for BM acceptance (table by author)}
	\label{elements_bi_acceptance}
\end{table}

According to the previous paragraphs the answers to the two proposed research questions would be as followed. First, a recommendation agent and a sandbox as elements of a BI platform can be used to increase the acceptance of benefits management. Second, these elements of a BI platform fulfill at least subjective selected meta-requirements and design principles for the implementation or realization of BM.

\chapter{Conclusion and Outlook}
\label{conclusion_outlook}
The paper starts with an introduction in the Cranfield approach of benefits management. Then the term of business intelligence is classified in the context of information systems. After presenting ongoing challenges in business intelligence possible meta-requirements and derived design principles are described to engage those challenges. The next section provides results of literature review about the acceptance of BM on one hand and meta-requirements and design principles for a successful realization of BM on the other hand. After that possible relations between elements of BI platform and the acceptance variables and design principles of BM are discussed.
\newline\newline
The suggested overlapping and relations are the foundation for the proposal that investing in business intelligence leads to an higher acceptance of benefits management tasks itself, due to the fact that some business intelligence artefacts (reports statistics and recommendation as well as sandboxes) are considered as BM artefacts. This is because they are supporting the realization of long-term benefits for the BI-investments, especially the steps for ``reviewing and evaluating the benefits'' and ``establishing of potential for further benefits''. The current shortcomings of this proposal are obvious. First, it exists no empirical evidence and the overlapping and relations aren't verifiable at this moment. Second, the exploration of possible overlapping and relations aren't complete, because there are existing more variables for BM acceptance and more meta-requirements and derived design principles for BM than mentioned in section \ref{bi_elements_higher_acceptance}. Third, the proposal concentrate on the the last two steps of the benefits management process, but it isn't guaranteed that the other steps can also be supported. Nevertheless this could be a good start for a higher acceptance of benefits management by realizing long-term benefits for BI systems and then other types of information systems. 
\newline\newline
Further research could concentrate on the verification of the assumptions, which where are made about the overlapping and relations, by doing surveys and tests in controlled environments. It seems also possible to complete a subjective comparison between the determinants and design principles for BM acceptance on the one hand and the design principles for BI platforms on the other hand. For practitioners it could be a reasonable start to use business intelligence platform to introduce and develop benefits management for and in their companies, considering the current and upcoming challenges in that context.


